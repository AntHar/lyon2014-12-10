\documentclass[xcolor=dvipsnames, 12, c, usepdftitle=false]{beamer}
% \documentclass{beamer}

\usepackage[frenchb]{babel}
\usepackage[T1]{fontenc}
% \usepackage[latin1]{inputenc}
\usepackage[utf8x]{inputenc}
%\usepackage{default}

\usetheme{Warsaw}
% \usetheme{PaloAlto}
%\usecolortheme[named=BurntOrange]

\title[JJCR \& GDR Robotique '14]{Retour sur JJCR \& GdR Robotique 2014}
% \title{JJCR \& GdR Robotique 2014}
% \author[A. Harnist]{Antony Harnist \\ \texttt{antony.harnist@grenoble-inp.gipsa-lab.fr}}
\author[A. Harnist]{Antony Harnist}
\institute[GIPSA-lab]{Grenoble Images Parole Signal Automatique}
\date[18/11/2014]{Mardi 18 Novembre 2014}
% \date{\today}
%\email{antony.harnist@grenoble-inp.gipsa-lab.fr}

% \section{Section 1}
% \subsection{Sous section 1 de la Section 1}
  % \begin{frame}
    % Ma première page !
  % \end{frame}
% \subsection{Sous section 2 de la Section 1}
  % \begin{frame}
    % Et maintenant ma deuxième page !
  % \end{frame}
% 
% \section{Section 2}
% \subsection{Sous section 1 de la Section 2}
  % \begin{frame}
    % Voici ma troisième page, elle appartient à ma deuxième section ! :) 
  % \end{frame}
% \subsection{Sous section 2 de la Section 2}
  % \begin{frame}
    % Et celle là c'est la deuxième page, mais de ma deuxième section. 
  % \end{frame}
 
%%%%%%%%%%%%%%%%%%%%%
% Guillaume Tréhard : Localisation et Cartographie Simultanées et Crédibiliste
%	Utilisation de grille; lidar -> grille polaire en forme de cône; modification carte polaire en coordonées cartésiennes; matching de la carte(t_{k-1}) avec carte(t_{k}); suppression des conflits
%
%%%%%%%%%%%%%%%%%%%%%
% Louis-Charles Caron : RGB-D object recognition for segmantic mapping
%	apprentissage des objets très supervisé; utilisatiion de détecteur de texture (SIFT),
%								 détecteurs de couleurs (HSV (utilisation de l'histogramme de la teinte) et RGB (utilisation et normalisation de l'histogramme)), 
%								 détecteur de forme.
%	Utilisations des différents détecteurs en entré d'un réseau de neuronnes artificiels qui en sortie retourne une/deux classes.
%	Segmentation de ce qui est percu par la caméra.
%	Effectue un moyennage des différents scores pour des objets superposés sur la carte!
%%%%%%%%%%%%%%%%%%%%%
% Gilles Tagne : Planification de trajectoires et commande pour la navigation des véhicules autonomes (POSTER)
% 	Méthode des tentacules. Permet d'effectuer le suivi d'une personne de manière plus fuide 

	 
\begin{document}

\begin{frame}
\maketitle			% équivalent à \titlepage 
\end{frame}

\begin{frame}[c, label=G. Trehard]
  \frametitle{Localisation et Cartographie Simultanées et Crédibiliste}
  \framesubtitle{Guillaume Tréhard}
  Catégorie : Cartographie. \\
  Méthode : Théorie des croyances.\\

  \begin{itemize}
    \item Utilisation de grille;
    \item Lidar -> grille polaire;
    \item Transformation grille polaire en coordonnées cartésiennes;
    \item Matching de la carte($t_{k-1}$) avec carte($t_{k}$);
    \item Suppression des conflits.
  \end{itemize}
\end{frame}
 
\begin{frame}[c, label=LC. Caron]
  \frametitle{RGB-D object recognition for segmantic mapping}
  \framesubtitle{Louis-Charles Caron}
  Catégorie : Cartographie et classification.\\
  Méthode : Réseau de neurones artificiels.\\
  Utilisation de multiples descripteurs:
      \begin{itemize}
	  \item Descripteur de texture (SIFT);
	  \item Descripteur de couleurs: histogramme de la teinte, histogramme normalisé RGB; 
	  \item Descripteur de formes;
      \end{itemize}
  Utilisations d'un réseau de neuronnes artificiels.\\
  Ségmentation.\\
  Effectue un moyennage des différents scores pour des objets superposés sur la carte!
\end{frame}


\begin{frame}[c, label=G. Tagne]
  \frametitle{Planification de la trajectoire et commande pour la navigation des véhicules autonomes (poster)}
  \framesubtitle{Gilles Tagne}
  Catégorie : Cartographie\\
  Méthode des tentacules
\end{frame}

% \begin{frame}
%   \tableofcontents
% \end{frame}

% \begin{frame}[label=slideTitre] 	%[c|t|b] center, top bottom align
% 					%[plain] suppress headlines, footlines and sidebars. Usefull to show large pictures
% 					%[allowframebreaks] permet de séparer un slide en deux si trop grand
% 
%   \frametitle{Récapitulatif JJCR et GdR Robotique 2014}
%   \framesubtitle{exemple de sous-tittre}
%   Corps du slide
% \end{frame}
% 


% 
% \begin{frame}

%   \begin{itemize}
%     \item Language used by Beamer: L\uncover<2->{A}TEX
%     \item Language used by Beamer: L\only<2->{A}TEX
%   \end{itemize}
% \end{frame}


\end{document}
