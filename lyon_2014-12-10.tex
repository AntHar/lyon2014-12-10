\documentclass[usepdftitle=false, xcolor=dvipsnames, 12, c]{beamer} %
%\usepackage{} %=dvipsnames, 12, c, usepdftitle=false]{beamer}

\usepackage[utf8]{inputenc}
\usepackage[french]{babel}
\usepackage[T1]{fontenc}
\usepackage{amssymb, amsmath}
\usepackage{graphicx}
\graphicspath{ {./images/} }
\usefonttheme{professionalfonts}
\usetheme{Warsaw}
\expandafter\def\expandafter\insertshorttitle\expandafter{\insertshorttitle\hfill\insertframenumber}%\,/\,\inserttotalframenumber}
%\usecolortheme{dolphin}
%\setbeamertemplate{headline}{}
%\setbeamertemplate{footline}[frame number]
\beamertemplatenavigationsymbolsempty       % ou \setbeamertemplate{navigation symbols}{}
%\definecolor{orange}{RGB}{255,127,0} %exemple couleur    utilisation     \textcolor{nom_couleur}{texte en couleur}
%\usepackage[]{beamerthemegipsa} % options: nogradient nobackground
%\usetheme{Gipsa}

\title[ARC6 : Robots compagnons]{\Large \textbf{Surveillances de personnes en situation de fragilité par une équipe de robots compagnons}}
\author[Antony Harnist]{\Large \textbf{Antony HARNIST}\\ \small antony.harnist@gipsa-lab.grenoble-inp.fr\\ \vspace{1cm}
Encadrement: \textbf{Michèle ROMBAUT} et \textbf{Didier COQUIN}}
\institute{\Large \textbf{GIPSA}\normalsize-lab et \Large \textbf{LISTIC} }
\logo{\includegraphics[height=15mm]{logoListic.png} ~ \includegraphics[height=15mm]{logo-gipsa-bas-BD.jpg}}
\date[10/12/2014]{Mercredi 10 Décembre 2014}


% pour pllus d'info : http://tex.stackexchange.com/questions/82794/removing-page-number-from-title-frame-without-changing-the-theme
\let\otp\titlepage
\renewcommand{\titlepage}{\otp\addtocounter{framenumber}{-1}}


\begin{document}

%%%%%%%%%%%%%%%% Diapo Titre
\begin{frame}[plain, label=titre]   %[plain]
    \maketitle    %\titlepage    %
\end{frame} 


% \begin{frame}
%     \frametitle{Sommaire}
%     \tableofcontents%[currentsection] %currentsubsection,hideallsubsections,hideothersubsections,pausesections,pausesubsections
% \end{frame}

% \AtBeginSection[]{
%     \frame{
%         \frametitle{}
%         \tableofcontents[currentsection, hideothersubsections, pausesubsections]
%     }
% }

\section{Laboratoires}
\begin{frame}[label=laboratoires]
    \frametitle{Laboratoires} % / Institutions}
    \textbf{GIPSA}-lab, équipe AGPIG \\
        (Architecture-Géométrie, Perception, Images, Gestures)\\
        11 Rue des Mathématiques, BP46, 38402 Saint Martin d’Hères\\
        Michèle ROMBAUT	\small(michele.rombaut@gipsa-lab.grenoble-inp.fr)\\
    \vspace{2em}
    \textbf{LISTIC}, équipe FIAD \\
        (Fusion d'Informations pour l'Aide à la Décision)\\
        5 Chemin de Bellevue, BP 80439, 74944 Annecy-le-Vieux Cedex\\
        Didier COQUIN	\small(didier.coquin@univ-savoie.fr)\\
\end{frame} 


% \begin{frame}[label=objectifs, t]
% \frametitle{Objectifs}
% Architecture multi-robots pour l'analyse de la situation d'une personne.\\
% Ensemble de robots vu comme un système de perception multi-capteurs adaptable.\\
% \end{frame}

\section{Objectifs}
\subsection{Contexte}
\begin{frame}[label=objContexte, t]
\frametitle{Objectifs}
    Architecture multi-robots pour l'analyse de la situation d'une personne.\\
    Ensemble de robots vu comme un système de perception multi-capteurs adaptable.\\
    \vspace{1em}
    Contexte
    \begin{itemize}
        \item[] personnes en environnement intérieur;
        \item[] analyse de la situation $\rightarrow$ pas d'intéraction;
        \item[] ne pas gêner la personne;
    \end{itemize}
\end{frame}

\setcounter{framenumber}{1}
\subsection{Sorties}
\begin{frame}[label=objSorties, t]
\frametitle{Objectifs}
    Architecture multi-robots pour l'analyse de la situation d'une personne.\\
    Ensemble de robots vu comme un système de perception multi-capteurs adaptable.\\
    \vspace{1em}
    Sorties : situation de la personne
    \begin{itemize}
        \item[] localisation dans un carte dynamique;
        \item[] situation physique (debout, assise, etc.);
        \item[] situation comportementale (marche, parle, regarde la télévision, etc.);
    \end{itemize}
\end{frame}

\setcounter{framenumber}{1}
\subsection{Multi-robots}
\begin{frame}[label=objMultirobots, t]
\frametitle{Objectifs}
    Architecture multi-robots pour l'analyse de la situation d'une personne.\\
    Ensemble de robots vu comme un système de perception multi-capteurs adaptable.\\
    \vspace{1em}
    Multi-robots : augmentation des capacités perceptuelles
    \begin{itemize}
        \item[] nécessité d'une coopération;
    \end{itemize}
    \begin{figure}[p]
        \includegraphics[height=2cm]{nao.jpg} ~~~~ \includegraphics[height=2cm]{qbo_xtion.jpg}
    \end{figure}
\end{frame}

\section{Problématiques scientifiques}
\begin{frame}[label=pbScientifiques]
\frametitle{Problématiques scientifiques}
\begin{itemize}
    \item fusion d'informations provenant des systèmes de perception des robots;
    \item communication entre robots ; protocole + message transmis;
    \item stratégie de déplacement des robots pour accroître la connaissance sur la situation;
    \item modélisation de la situation;
\end{itemize}
\end{frame}


\section{Les pistes de recherches}
\begin{frame}[label=pistesRecherches]
\frametitle{Les pistes de recherches}
    \begin{itemize}
        \item fusion d'informations 
        \begin{itemize}
            \item[] hétérogènes (sons, images, images de profondeur), connaissances a priori, incertaines, incomplètes ou absentes.
        \end{itemize}
        \item détection et analyse du comportement de personnes 
        \item stratégie de déplacement des robots (améliorer la perception)
        \item modélisation de la situation
        \begin{itemize}
            \item[] cartes statiques/dynamiques, évolutives, ect.
        \end{itemize}
        \item informations à échanger ou partager
    \end{itemize}
\end{frame}


\end{document}
