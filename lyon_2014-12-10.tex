\documentclass[usepdftitle=false, xcolor=dvipsnames, 12, c]{beamer}
%\usepackage{} %=dvipsnames, 12, c, usepdftitle=false]{beamer}

\usepackage[utf8]{inputenc}
\usepackage[french]{babel}
\usepackage[T1]{fontenc}
\usepackage{amssymb, amsmath}
\usepackage{graphicx}
\graphicspath{ {./images/} }
\usefonttheme{professionalfonts}
\usetheme{Warsaw}
\setbeamertemplate{headline}{}
\setbeamertemplate{footline}[frame number]
\beamertemplatenavigationsymbolsempty       %\setbeamertemplate{navigation symbols}{}

%\usepackage[]{beamerthemegipsa} % options: nogradient nobackground
%\usetheme{Gipsa}

% \title[Robots compagnons]{Surveillances de personnes en situation de fragilité par une équipe de robots compagnons}
% \author[Antony HARNIST]{\Large Antony HARNIST }% \\\small \textbf{antony.harnist@gipsa-lab.grenoble-inp.fr}}
% % %\institute{\Large\textbf{GIPSA}-lab \\% \titlegraphic{\includegraphics{../logo/logo-gipsa-bas-BD.jpg}}\\
% % % \normalsize 11 rue des Mathématiques, F-38420, Grenoble\\
% % % \vspace{1em}
% % % \Large\textbf{LISTIC} \\% \titlegraphic{\includegraphics{../logo/logoListic.png}}\\
% % % \normalsize 5 Chemin de Bellevue, F-74944, Annecy-le-Vieux\\
% % % %\titlegraphic{\includegraphics{logoListic.png} \includegraphics{logo-gipsa-bas-BD.jpg}}
% % \institute{\Large \textbf{LISTIC} \& \Large \textbf{gipsa}-lab \normalsize}
% % \logo{\includegraphics[height=1cm]{../images/logoListic.png} ~ \includegraphics[height=1cm]{../images/logo-gipsa-bas-BD.jpg}}
% % \date[10/12/2014]{Mercredi 10 Décembre 2014}

\title[ARC6 : Robots compagnons]{\Large \textbf{Surveillances de personnes en situation de fragilité par une équipe de robots compagnons}}
\author{\Large \textbf{Antony HARNIST}\\ \small antony.harnist@gipsa-lab.grenoble-inp.fr\\ \vspace{1cm}
Encadrement: \textbf{Michèle ROMBAUT} et \textbf{Didier COQUIN}}
\institute{\Large \textbf{GIPSA}\normalsize-lab et \Large \textbf{LISTIC} }
\logo{\includegraphics[height=15mm]{logoListic.png} ~ \includegraphics[height=15mm]{logo-gipsa-bas-BD.jpg}}
\date[10/12/2014]{Mercredi 10 Décembre 2014}

%%%%%%%%%%%%%%%% Début du document
\begin{document}

%%%%%%%%%%%%%%%% Diapo Titre
\begin{frame}[label=titre]   %[plain]
\titlepage    %\maketitle
\end{frame} 

% \begin{frame}
%     \frametitle{Sommaire}
%     \tableofcontents%[currentsection] %currentsubsection,hideallsubsections,hideothersubsections,pausesections,pausesubsections
% \end{frame}

% \AtBeginSection[]{
%     \frame{
%         \frametitle{}
%         \tableofcontents[currentsection, hideothersubsections, pausesubsections]
%     }
% }

\section{Laboratoires}
\begin{frame}[label=laboratoires]
\frametitle{Laboratoires} % / Institutions}
\textbf{GIPSA}-lab, équipe AGPIG \\
(Architecture-Géométrie, Perception, Images, Gestures)\\
11 Rue des Mathématiques, BP46, 38402 Saint Martin d’Hères\\
Michèle ROMBAUT	\small(michele.rombaut@gipsa-lab.grenoble-inp.fr)\\
\vspace{2em}
\textbf{LISTIC}, équipe FIAD \\
(Fusion d'Informations pour l'Aide à la Décision)\\
5 Chemin de Bellevue, BP 80439, 74944 Annecy-le-Vieux Cedex\\
Didier COQUIN	\small(didier.coquin@univ-savoie.fr)\\
\end{frame} 


\section{Objectifs}
\begin{frame}[label=objectifs]
\frametitle{Objectifs}
Architecture multi-robots pour l'analyse de la situation d'une personne.\\
Ensemble de robots vu comme un système de perception multi-capteurs adaptable.\\
\end{frame}

\subsection{Contexte}
\begin{frame}[label=objContexte]
\frametitle{Objectifs}
Contexte
\begin{itemize}
    \item[] personnes en environnement intérieur;
    \item[] analyse de la situation $\rightarrow$ pas d'intéraction;
    \item[] ne pas gêner la personne;
\end{itemize}
\end{frame}

\subsection{Sorties}
\begin{frame}[label=objSorties]
\frametitle{Objectifs}
Sorties : situation de la personne
    \begin{itemize}
        \item[] localisation dans un carte dynamique;
        \item[] situation physique (debout, assise, etc.);
        \item[] situation comportementale (marche, parle, regarde la télévision, etc.);
    \end{itemize}
\end{frame}

\subsection{Multi-robots}
\begin{frame}[label=objMultirobots]
\frametitle{Objectifs}
    Mulit-robots : augmentation des capacités perceptuelles
    \begin{itemize}
        \item[] nécessité d'une coopération;
    \end{itemize}
    \begin{figure}[p]
        \includegraphics[height=2cm]{nao.jpg} ~~~~ \includegraphics[height=2cm]{qbo_xtion.jpg}
    \end{figure}
\end{frame}

\section{Problématiques scientifiques}
\begin{frame}[label=pbScientifiques]
\frametitle{Problématiques scientifiques}
\begin{itemize}
    \item fusion d'informations provenant des systèmes de perception des robots;
    \item communication entre robots ; protocole + message transmis;
    \item stratégie de déplacement des robots pour accroître la connaissance sur la situation;
    \item modélisation de la situation;
\end{itemize}
\end{frame}

\section{Les pistes de recherches}
\begin{frame}[label=pistesRecherches]
\frametitle{Les pistes de recherches}
    \begin{itemize}
        \item fusion d'informations 
        \begin{itemize}
            \item[] hétérogènes (sons, images, images de profondeur), connaissances a priori, incertaines, incomplètes ou absentes.
        \end{itemize}
        \item détection et analyse du comportement de personnes 
        \item stratégie de déplacement des robots (améliorer la perception)
        \item modélisation de la situation
        \begin{itemize}
            item[] cartes statiques/dynamiques, évolutives, ect.
        \end{itemize}
        \item informations à échanger ou partager
    \end{itemize}
\end{frame}

\end{document}
